%% Elsevier 'elsarticle' template — Structures (review mode)
\pdfminorversion=7 % allow inclusion of up to PDF 1.7 graphics
\documentclass[review,12pt]{elsarticle} % ilk gönderimde review + tek sütun
\journal{Structures}
\biboptions{square}

%% Packages
\usepackage[english]{babel}
\usepackage{amsmath,amssymb,mathtools}
\usepackage{graphicx}
\usepackage{siunitx}
\usepackage{microtype}
\makeatletter
\AtBeginDocument{\@ifundefined{MT@orig@showhyphens}{}{\let\showhyphens\MT@orig@showhyphens}}
\makeatother
\usepackage{lineno}
\usepackage{booktabs}
\usepackage{tabularx}
\usepackage{array}
\usepackage{ragged2e} % daha iyi satır sonu için (opsiyonel)
\usepackage{placeins}



%% siunitx
\sisetup{
mode = match,
propagate-math-font = true,
reset-math-version = false,
reset-text-family = false,
reset-text-series = false,
reset-text-shape = false,
text-family-to-math = true,
text-series-to-math = true,
per-mode = symbol,
range-units = single,
range-phrase = --,
exponent-product = \times
}

%% Microtype
\microtypesetup{expansion,protrusion}
\emergencystretch=3em
\allowdisplaybreaks[1]

% --- HYPERREF EN SONDA ---
\usepackage{hyperref}
\makeatletter
\pdfstringdefDisableCommands{%% disable front-matter commands in PDF strings
\def\corref#1{}
\def\cortext#1{}
\def\cnotenum#1{}
\def\@corref#1{}
}
\makeatother
\newcolumntype{L}{>{\RaggedRight\arraybackslash}X}
\providecommand*{\theHpage}{\thepage}

\begin{document}
\pagenumbering{roman}
\renewcommand*{\theHpage}{FR\roman{page}}

\section{Methods: Physical \& Numerical Modelling}

\begin{figure}[t]
  \centering
  \includegraphics[width=0.6\textwidth]{sistemcs.pdf}
  \caption{Schematic diagram of the $n$-storey shear frame equipped with a roof-level fluid viscous damper (FVD).}
  \label{fig:FVDschematic}
\end{figure}

\begin{sloppypar}
Figure~\ref{fig:FVDschematic} illustrates the $n$-DOF shear frame equipped with a roof-level FVD considered in this study. This work utilizes a high-fidelity numerical model coupling the $n$-DOF structural dynamics of a shear frame with the nonlinear hydro--thermal behavior of the FVDs. We intentionally move beyond simplified phenomenological laws (e.g., $F=C|v|^{\alpha}$). Instead, the FVD sub-model is mechanistic, incorporating: (i) orifice hydraulics with a Reynolds-dependent discharge coefficient $C_d(\mathrm{Re})$ and jet-loss scaling $\Delta p\!\propto\!Q^2$ \citep{Yau2017,Hutagalung2019,Osman2019}; (ii) a physics-based cavitation limiter implemented via a differentiable soft--minimum between the jet candidate and an effective vapour threshold, consistent with orifice cavitation inception criteria \citep{Davoudi2021}; and (iii) a two-node thermal block that couples temperature to viscosity (and density), so that both $c_{\mathrm{lam}}(T)$ and $C_d(\mathrm{Re}(T))$ respond to heating \citep{Yau2017,Zhong2020}.
\end{sloppypar}

\noindent\textit{Rationale.} Classical $F=C|v|^{\alpha}$ laws lump laminar and jet mechanisms and cannot track pressure, cavitation, and thermal safety under pulse-type demands. For sharp-edged short orifices in the relevant $\mathrm{Re}$ range, $\Delta p$ follows a quadratic $Q$-law with $C_d$ varying with $\mathrm{Re}$ and geometry \citep{Yau2017,Hutagalung2019,Osman2019}; keeping linear viscous losses in a \emph{parallel} branch avoids double counting inside the orifice drop. Cavitation onset places a physical cap tied to vapour pressure and local accelerations \citep{Davoudi2021}, while practical FVD design/qualification likewise imposes explicit pressure/capacity limits carried forward into our QC metrics \citep{Zhong2020,Acquaro2020SDEE,Song2016ICMS}.

\par
This integration results in a stiff, coupled ODE system that must advance both structural DOFs and damper temperatures simultaneously[cite: 331]. For numerical robustness and reproducibility, this system is solved using MATLAB's \texttt{ode15s} integrator with strict tolerances ($\text{RelTol}=10^{-3}$ and $\text{AbsTol}=10^{-6}$)[cite: 131, 134, 150, 163]. All performance and safety metrics (PFA, IDR, pressure, temperature) are subsequently evaluated over the significant duration ($t_5-t_{95}$) of the ground motion[cite: 332, 372].


\subsection{Structural model}\label{sec:structural_model}
We idealise the building as a planar, 10-storey \emph{shear-type} frame with one lateral degree of freedom (DOF) per floor. Let $\boldsymbol{x}(t)\!\in\!\mathbb{R}^{10}$ denote \emph{relative} floor displacements (with respect to the moving base) and let $\boldsymbol{r}=\boldsymbol{1}\in\mathbb{R}^{10}$ be the base-excitation influence vector. The equations of motion are
\begin{equation}
  \boldsymbol{M}\,\ddot{\boldsymbol{x}}(t)
  + \boldsymbol{C}_0\,\dot{\boldsymbol{x}}(t)
  + \boldsymbol{K}\,\boldsymbol{x}(t)
  + \boldsymbol{f}_{\mathrm{d}}(t)
  \;=\;
  -\,\boldsymbol{M}\,\boldsymbol{r}\,a_g(t),
  \label{eq:eom}
\end{equation}
where $\boldsymbol{M}$, $\boldsymbol{C}_0$, and $\boldsymbol{K}$ are the mass, structural damping, and stiffness matrices; $\boldsymbol{f}_{\mathrm{d}}(t)$ collects the device-induced nodal forces. All symbols use SI. The full symbol list, assembly details, and the specific parameters for this 10-DOF benchmark structure are consolidated in Appendix~\ref{app:derived}.

\paragraph{Storey kinematics and device coupling}
Let $\boldsymbol{B}\in\mathbb{R}^{9\times 10}$ be the first-difference (incidence) operator, so that storey drift and drift-rate read
\begin{equation}
  \Delta\boldsymbol{x}(t)=\boldsymbol{B}\,\boldsymbol{x}(t),\qquad
  \Delta\dot{\boldsymbol{x}}(t)=\boldsymbol{B}\,\dot{\boldsymbol{x}}(t).
  \label{eq:drift}
\end{equation}
If $\boldsymbol{f}_{\mathrm{story}}(t)\in\mathbb{R}^{9}$ stacks the forces delivered by between-floor devices, the corresponding nodal contribution assembles as
\begin{equation}
  \boldsymbol{f}_{\mathrm{d}}(t)=\boldsymbol{B}^{\mathsf T}\,\boldsymbol{f}_{\mathrm{story}}(t).
  \label{eq:fd_assembly}
\end{equation}

\paragraph{Response measures (Arias window)}
Absolute floor accelerations are
\begin{equation}
  \boldsymbol{a}_{\mathrm{abs}}(t)=\ddot{\boldsymbol{x}}(t)+\boldsymbol{r}\,a_g(t),
  \label{eq:aabs}
\end{equation}
and the roof peak floor acceleration (PFA) is evaluated on the Arias-intensity window $[t_5,t_{95}]$ as
\begin{equation}
  \mathrm{PFA}=\max_{t\in[t_5,t_{95}]}\big|\,\boldsymbol{e}_{10}^{\mathsf T}\boldsymbol{a}_{\mathrm{abs}}(t)\,\big|.
  \label{eq:PFA}
\end{equation}
For uniform storey height $h$, the maximum inter-storey drift ratio (IDR) is
\begin{equation}
  \mathrm{IDR}=\max_{t\in[t_5,t_{95}]}\left\| \frac{\Delta\boldsymbol{x}(t)}{h} \right\|_{\infty}.
  \label{eq:IDR}
\end{equation}

\noindent\textit{Notes.} The standard tri-diagonal assembly of $(\boldsymbol{K},\boldsymbol{C}_0)$ from storey parameters and the modal notation used for intensity scaling (e.g., $T_1$) are deferred to Appendix~\ref{app:derived}; time-domain integration uses Eq.~\eqref{eq:eom} directly.

\subsection{Damper physics}\label{sec:damper_physics}

\begin{figure}[t]
  \centering
  \includegraphics[width=0.9\linewidth]{ViscousDamper_CAD.pdf}
  \caption{Schematic of the high-fidelity fluid viscous damper (FVD) model. Key components include the piston head containing orifices, which separates Chamber~1 and Chamber~2; the working fluid is a compressible silicone oil \citep{wacker:2002:siliconefluids,tamson:2019:siliconoil}, and a parallel coil spring represents $k_{sd}$.}

  \label{fig:damper_schematic}
\end{figure}

The storey device is modelled as a \emph{parallel} combination of a linear elastic branch and a hydraulic orifice branch, as depicted in Figure~\ref{fig:damper_schematic}. With the storey drift $\Delta x(t)$ and drift rate $\Delta \dot x(t)$ defined as
\begin{equation}
\Delta x(t)=x_{i+1}(t)-x_i(t),\qquad 
\Delta \dot x(t)=\dot x_{i+1}(t)-\dot x_i(t),
\label{eq:dx-dv}
\end{equation}
and with $A_p$ the piston area and $A_o$ the \emph{total} orifice area ($A_o=n_{\mathrm{orf}}\pi d_o^2/4$), the storey force splits into three components:
\begin{equation}
F_{\mathrm{story}}(t)=F_{\mathrm{elastic}}(t)+F_{\mathrm{laminar}}(t)+F_{\mathrm{orifice}}(t),
\label{eq:Fstory}
\end{equation}
where the elastic and laminar contributions are
\begin{equation}
F_{\mathrm{elastic}}(t)=k_{sd}\,\Delta x(t),\qquad
F_{\mathrm{laminar}}(t)=c_{\mathrm{lam}}(T)\,\Delta \dot x(t).
\label{eq:Felam}
\end{equation}
Here $k_{sd}$ lumps the hydraulic/cylinder compliance in series with any coil spring; its full geometric definition is deferred to the Appendix.

\paragraph{Kinematics, layout, and multiplicity}
Positive $\Delta x$ increases the upper chamber volume and decreases the lower one; algebraic flow is taken positive from upper to lower. 

This study models a 2D planar shear-frame, assuming one equivalent damper element ($n_\parallel=1$) is placed at each of the 9 inter-storey levels. The optimization (Section~\ref{sec:opt_ga}) seeks a single, uniform design (i.e., one set of parameters) to be applied to all 9 devices. This $n_\parallel=1$ approach represents the total \emph{equivalent} hydraulic circuit for that storey. In a practical application where multiple physical cylinders are installed in parallel, their contributions (e.g., $A_p$, $A_o$, $hA$) are assumed to be lumped into this single optimized equivalent element, thus avoiding parameter redundancy. For a regular-plan building, the same damper design would be assumed for the orthogonal (Y) direction.

\subsubsection*{Orifice hydraulics: jet loss with $C_d(\mathrm{Re})$ (no linear drop inside $\Delta p$)}

To establish a basis for the safety metric $(Q/Q_{\mathrm{cap}})_{95}$ and to normalize the flow saturation, we define the theoretical orifice-flow capacity under a nominal design pressure cap $\Delta p_{\mathrm{cap}}$:
\begin{equation}
Q_{\mathrm{cap}} = C_d^\infty A_o \sqrt{ \frac{2 \Delta p_{\mathrm{cap}}}{\rho(T)} }.
\label{eq:Qcap_def}
\end{equation}
Here, the design pressure cap is defined as
$\Delta p_{\mathrm{cap}}=\phi\,p_{\mathrm{work}}$ with $\phi=1.5$,
where $p_{\mathrm{work}}$ is the manufacturer-rated maximum working pressure \citep{Eiga2024}.
In the absence of a project-specific rating, we adopt a conservative
$\Delta p_{\mathrm{cap}}=\SI{20}{MPa}$, which lies in the lower half of the
typical operating range reported for seismic FVD orifices (14-69 MPa).
This choice provides proof-level headroom consistent with hydraulic
pressure-testing practice ($\ge 1.5\times$ working pressure) \citep{Eiga2024}.

The saturated flow used in the jet-loss term is then
\begin{equation}
Q_{\mathrm{sat}}(\Delta\dot x)=
Q_{\mathrm{cap}}\;\tanh\!\Big(\frac{A_p}{Q_{\mathrm{cap}}}\sqrt{\Delta\dot x^{\,2}+v_\varepsilon^{\,2}}\Big),
\qquad v_\varepsilon>0.
\label{eq:Qsat}
\end{equation}

Consistent with single–phase orifice behavior, only the quadratic (jet) head loss contributes to the orifice pressure candidate,
\begin{equation}
\Delta p_{\mathrm{kv}}(Q_{\mathrm{sat}},T)=\frac{\rho(T)\,Q_{\mathrm{sat}}\,|Q_{\mathrm{sat}}|}{2\,[C_d(\mathrm{Re})\,A_o]^2},
\label{eq:dp-kv}
\end{equation}
in line with the measured $\Delta p \propto Q^2$ (equivalently $Q\propto\sqrt{\Delta p}$) relationships for sharp/short orifices at relevant Reynolds numbers \citep{Cioncolini2018,Abd2019,Golijanek-Jdrzejczyk2023}. 
\textit{Linear viscous losses are represented explicitly by the parallel laminar branch} $c_{\mathrm{lam}}(T)$ \textit{in Eq.~\eqref{eq:Felam}; embedding an additional linear drop inside} $\Delta p$ \textit{would double-count the same mechanism}.

The Reynolds number and the discharge–coefficient transition are defined as
\begin{equation}
\mathrm{Re}=\frac{\rho(T)\,|Q_{\mathrm{sat}}|\,d_o}{\mu(T)\,A_o},\qquad
C_d(\mathrm{Re})=
C_d^\infty-\frac{C_d^\infty-C_d^0}{1+(\mathrm{Re}/\mathrm{Re}_0)^{p_{\exp}}},
\label{eq:Cd-Re}
\end{equation}
with $C_d$ clamped to admissible bounds. While the literature shows that $C_d$–Re behavior and the onset of a high–Re plateau depend on geometry and operating conditions \citep{laMorena2018,Lee2018,Ahmed2020,Gao2019}, in this study we hold the transition scale fixed at $\mathrm{Re}_0=\num{1000}$ to avoid over-parameterization and identifiability issues in the 12-variable design. The fixed value and justification are listed in Appendix Table~\ref{tab:fixed-values-phys}.


\paragraph{Laminar equivalent (used \emph{outside} $\Delta p_{\mathrm{kv}}$)}
For diagnostics and the explicit viscous branch in Eq.~\eqref{eq:Felam}, the laminar network is represented by its equivalent flow resistance $R_{\mathrm{lam}}$. The resulting laminar coefficient $c_{\mathrm{lam}}$ follows directly from the standard hydraulic-to-mechanical equivalence $c_{\mathrm{lam}} = R_{\mathrm{lam}} A_p^2$:
\begin{equation}
R_{\mathrm{lam}}(T)=\frac{128\,\mu(T)\,L_{\mathrm{ori}}}{\pi\,d_o^4}\,\frac{1}{n_{\mathrm{orf}}},\qquad
c_{\mathrm{lam}}(T)=R_{\mathrm{lam}}(T)\,A_p^2,
\label{eq:Rlam-clam}
\end{equation}
so that $F_{\mathrm{laminar}}=c_{\mathrm{lam}}(T)\,\Delta\dot x$. Note that $R_{\mathrm{lam}}$ does \emph{not} enter Eq.~\eqref{eq:dp-kv}; there is no linear drop inside $\Delta p$.

\subsubsection*{Cavitation limiter (physical model)}\label{sec:damper_cav}
Cavitation is handled by applying a smooth lower bound to the orifice pressure. An upstream pressure proxy follows from the \emph{elastic} term relative to ambient:
\begin{equation}
p_{\uparrow}(t)=p_{\mathrm{amb}}+\frac{|F_{\mathrm{elastic}}(t)|}{A_p}.
\label{eq:pup}
\end{equation}
Comparing $p_{\uparrow}$ to an effective vapour threshold defines the cavitation-limited drop,
\begin{equation}
\Delta p_{\mathrm{cav}}(t)=\max\!\big\{\,\big(p_{\uparrow}(t)-p_{\mathrm{cav,eff}}\big)\cdot \mathrm{cav\_sf},\;0\,\big\},
\label{eq:dpcav}
\end{equation}
and the effective orifice pressure is the differentiable minimum
\begin{equation}
\Delta p_{\mathrm{eff}}=\operatorname{softmin}_{\varepsilon}\!\big(\Delta p_{\mathrm{kv}},\ \Delta p_{\mathrm{cav}}\big).
\label{eq:softmin}
\end{equation}
The cavitation fraction over the Arias window is the time ratio with $\Delta p_{\mathrm{kv}}>\Delta p_{\mathrm{cav}}$.

\paragraph{Cavitation threshold (physical basis).}
Room-temperature silicone oils exhibit extremely low vapour pressures (order of 1–2 Pa), and cavitation inception is strongly conditioned by dissolved gas and surface nuclei. Degassing protocols in silicone-oil Venturi experiments markedly alter inception behaviour, underscoring the role of gas content; diffusion-driven nucleation from surface nuclei further explains why inception can occur at technically relevant absolute pressures. Accordingly, we set an effective threshold $p_{\mathrm{cav,eff}}$ in the kPa range by applying a conservative safety factor above the pure vapour pressure, so that the limiter flags only physically plausible inception in civil-scale FVDs \citep{Croci2020ApplSci, GrossPelz2017JFM}.

\paragraph{Smooth minimum for numerical robustness.}
To keep the right–hand side differentiable for stiff time integration and gradient-based penalties, we use the log-sum-exp smooth minimum
\[
\operatorname{softmin}_{\varepsilon}(a,b)
= -\,\varepsilon \,\log\!\big(\mathrm{e}^{-a/\varepsilon}+\mathrm{e}^{-b/\varepsilon}\big),
\]
which monotonically approaches $\min(a,b)$ as $\varepsilon\!\downarrow\!0$, eliminating non-physical kinks while preserving the correct limiting behaviour \citep{Cuturi2013OTSoftmin}.

\paragraph{Orifice force with smoothed sign}
The hydraulic contribution to the storey force is
\begin{equation}
F_{\mathrm{orifice}}(t)=\Delta p_{\mathrm{eff}}(t)\,A_p\,
\frac{\Delta\dot x(t)}{\sqrt{\Delta\dot x^{\,2}(t)+v_\varepsilon^{\,2}}}.
\label{eq:Forf}
\end{equation}

\noindent\textit{Convention.} Linear (laminar) losses act \emph{only} through $c_{\mathrm{lam}}(T)$ in $F_{\mathrm{laminar}}$; the orifice pressure uses the \emph{jet} term with cavitation limiting, i.e. no linear drop is embedded in $\Delta p$.
\subsubsection*{Thermal block (coupled viscosity feedback)}
A compact two-node energy balance (oil $T_o$, steel/cylinder $T_s$) is advanced \emph{concurrently} with the structural state. This ensures that the temperature-dependent viscosity $\mu(T)$ is updated \emph{inside} the force evaluation, feeding back immediately into $c_{\mathrm{lam}}(T)$ and the Reynolds-dependent $C_d(\mathrm{Re}(T))$:
\begin{align}
C_o\,\dot T_o &= P_{\mathrm{loss}} - hA_{o\leftrightarrow s}(T_o-T_s) - hA_{o\leftrightarrow \mathrm{env}}(T_o-T_\infty), \label{eq:To}\\
C_s\,\dot T_s &= \phantom{P_{\mathrm{loss}}}\; +\,hA_{o\leftrightarrow s}(T_o-T_s) - hA_{s\leftrightarrow \mathrm{env}}(T_s-T_\infty). \label{eq:Ts}
\end{align}

\par
To clarify the thermal parameters, we link all three exchange coefficients in Eqs.~\eqref{eq:To}-\eqref{eq:Ts} to the single optimization variable $hA$ (total thermal conductance) listed in Table~\ref{tab:opt-vars}. We adopt a lumped model where $hA_{o\leftrightarrow s} = hA_{o\leftrightarrow \mathrm{env}} = hA_{s\leftrightarrow \mathrm{env}} = hA$; hence, no fixed baseline value is listed in the Appendix.

\noindent Consistent with the three-term force split, instantaneous mechanical power is the sum of the laminar and orifice contributions (no linear drop inside $\Delta p$):
\begin{equation}
P_{\mathrm{loss}}(t)=
\underbrace{c_{\mathrm{lam}}(T_o)\,\Delta\dot x^{\,2}(t)}_{P_{\mathrm{laminar}}}
+\underbrace{\Delta p_{\mathrm{eff}}(t)\,\big|Q_{\mathrm{sat}}(t)\big|}_{P_{\mathrm{orifice}}}
\;\ge 0 .
\label{eq:Ploss}
\end{equation}

\noindent\textit{Energy transfer and power split.}
The power split in Eq.~\eqref{eq:Ploss} follows standard hydraulic energetics:
the laminar branch dissipates mechanical power as
$P_{\mathrm{laminar}}=c_{\mathrm{lam}}(T_o)\,\Delta\dot x^{\,2}$,
whereas the orifice branch converts hydraulic power
$P=\Delta p_{\mathrm{eff}}\,|Q_{\mathrm{sat}}|$ into heat.
Summing these terms yields the total mechanical-to-thermal conversion that drives the two-node balance in Eqs.~\eqref{eq:To}–\eqref{eq:Ts}.
This is consistent with the mechanical-energy balance for short, sharp-edged orifices,
$Q=C_d(\mathrm{Re})\,A_o\sqrt{2\,\Delta p/\rho}$, hence $\Delta p\!\propto\!Q^{2}$; here $C_d$ is Reynolds-dependent at low to intermediate $\mathrm{Re}$ and approaches a plateau as $\mathrm{Re}$ increases \citep{Yau2017,Cheng2020,Ahmed2020}.

Temperature-dependent properties follow smooth laws:
\begin{equation}
\mu(T)=\mu_{\mathrm{ref}}\,\exp\!\big(b_\mu (T-T_{\mathrm{ref}})\big),\qquad
\rho(T)=\frac{\rho_{\mathrm{ref}}}{1+\alpha_\rho (T-T_{\mathrm{ref}})}.
\label{eq:props}
\end{equation}

\noindent\textit{Temperature-dependent properties.}
For PDMS silicone oils, the viscosity–temperature relation is well captured over engineering ranges 
(roughly $20$–$100\,^{\circ}$C and low-to-moderate shear) by log-linear/Arrhenius-type fits, which motivates 
the exponential form in Eq.~\eqref{eq:props}. We calibrate the fit around 
$T_{\mathrm{ref}}=\SI{25}{^\circ C}$ using manufacturer data for the specific oil grade (zero-shear region). 
The same dataset provides the density slope $\alpha_{\rho}$ used in Eq.~\eqref{eq:props}, ensuring that 
$\rho(T)$ and thus $\mathrm{Re}(T)$ and $C_d(\mathrm{Re}(T))$ respond consistently to heating 
\citep{Swallow2002,Venczel2021,Zhai2019,wacker:2002:siliconefluids,tamson:2019:siliconoil}.
\noindent The property slopes $b_\mu$, $\alpha_\rho$, and the optional $b_\rho$ used in Eq.~\eqref{eq:props} were obtained by least-squares fits to manufacturer data for PDMS silicone oils around $T_{\mathrm{ref}}=\SI{25}{^\circ C}$ \citep{tamson:2019:siliconoil,wacker:2002:siliconefluids}.

The accumulated dissipation reported on the Arias window is
\begin{equation}
E_{\mathrm{mech}}
=\int_{t_5}^{t_{95}} P_{\mathrm{loss}}(t)\,\mathrm{d}t
=\int_{t_5}^{t_{95}} \!\Big(c_{\mathrm{lam}}(T_o)\,\Delta\dot x^{\,2}(t)+\Delta p_{\mathrm{eff}}(t)\,\big|Q_{\mathrm{sat}}(t)\big|\Big)\,\mathrm{d}t .
\label{eq:Emech}
\end{equation}

\paragraph{Parameters (deferred)}
All fixed coefficients, bounds, and geometry for: (i) the discharge-coefficient transition $C_d(\mathrm{Re})$ (with fixed $\mathrm{Re}_0$); (ii) damper geometry and elastic/compressibility terms defining $k_{sd}$; (iii) the cavitation limiter; and (iv) thermal exchange and property slopes are consolidated in Appendix~\ref{app:derived} to keep the main text physics-first and reproducible. SI units are used throughout ($Q$ in \si{m^3\,s^{-1}}, $\Delta p$ in \si{Pa}, areas in \si{m^2}).

\subsection{Time integration}\label{sec:time_integration}
We advance the coupled problem with a \emph{stiff, variable–step} ODE scheme. The integrated state collects the structural DOFs and the two–node thermal pair, while damper hydraulics are evaluated \emph{algebraically} inside the right–hand side. With $\boldsymbol{x}(t)\in\mathbb{R}^{10}$ the relative floor displacements and $\boldsymbol{v}(t)=\dot{\boldsymbol{x}}(t)$, the first-order form is:
\[
\boldsymbol{z}(t)=\begin{bmatrix}\boldsymbol{x}(t)\\ \boldsymbol{v}(t)\\ T_o(t)\\ T_s(t)\end{bmatrix},\qquad
\dot{\boldsymbol{x}}=\boldsymbol{v},\qquad
\dot{\boldsymbol{v}}= -\,\boldsymbol{M}^{-1}\!\Big(\boldsymbol{C}_0\,\boldsymbol{v}+\boldsymbol{K}\,\boldsymbol{x}+\boldsymbol{f}_{\mathrm{d}}(t)\Big)\;-\;\boldsymbol{r}\,a_g(t),
\]
where $\boldsymbol{f}_{\mathrm{d}}(t)=\boldsymbol{B}^{\mathsf T}\boldsymbol{f}_{\mathrm{story}}(t)$ and $\boldsymbol{f}_{\mathrm{story}}$ is returned by the mechanistic device model using the instantaneous storey drift rate.

\paragraph{Device evaluation within the RHS}
At each call time, the damper routine uses $\Delta\dot x=\boldsymbol{B}\boldsymbol{v}$ and the current oil temperature to evaluate the saturated flow $Q_{\mathrm{sat}}$ (Eq.~\eqref{eq:Qsat}), the Reynolds number and $C_d(\mathrm{Re})$ (Eq.~\eqref{eq:Cd-Re}), the jet candidate $\Delta p_{\mathrm{kv}}$ (Eq.~\eqref{eq:dp-kv}), and the cavitation–limited $\Delta p_{\mathrm{eff}}$ (Eq.~\eqref{eq:softmin}). The storey force then assembles as the sum of the elastic term, the laminar branch, and the orifice contribution with sign smoothing (Eqs.~\eqref{eq:Felam}, \eqref{eq:Forf}).

\paragraph{Ground motion, integration, and response window}
The base acceleration $a_g(t)$ is accessed via linear interpolation. Given the stiff nature of the coupled system, integration is performed using MATLAB's \texttt{ode15s} solver, enforcing strict tolerances ($\text{RelTol}=10^{-3}$, $\text{AbsTol}=10^{-6}$) for robust reproducibility. The solution is sampled on the native record grid; however, all response metrics are computed only on the Arias–energy window $[t_5,t_{95}]$ to exclude start–up and coda effects.

\paragraph{Thermal coupling}
The two–node thermal block (Eqs.~\eqref{eq:To}–\eqref{eq:Ts}) is advanced concurrently and couples back through $\mu(T_o)$ and $\rho(T_o)$. The mechanical power driving the thermal balance is $P_{\mathrm{loss}}$ in Eq.~\eqref{eq:Ploss}; accumulated dissipation on $[t_5,t_{95}]$ is $E_{\mathrm{mech}}$ in Eq.~\eqref{eq:Emech}.

\paragraph{Initialization and masks}
The structure starts from rest, $\boldsymbol{x}(t_0)=\boldsymbol{0}$ and $\boldsymbol{v}(t_0)=\boldsymbol{0}$. Device multiplicity and storey–activity masks are applied algebraically (areas and linear coefficients scale with $n_\parallel$; inactive storeys are zeroed), preserving sparsity and robustness. Other numerical safeguards (e.g., soft-min sharpness, velocity smoothing) are detailed in Appendix~\ref{app:derived}.

\subsection{Ground Motion Processing and Scaling}
All analyses use a set of 10 recorded horizontal ground motions. Records are converted to SI, de-trended, and may be high-pass filtered (e.g., $f_c{=}0.05$ Hz)[cite: 340].

To ensure a consistent intensity measure (IM) across records, each accelerogram is scaled to a common target $\mathrm{IM}_\star$. This intensity is defined as the band-averaged geometric mean of the 5\%-damped pseudo-spectral acceleration, evaluated on $N$ linearly spaced periods within a band centered on the fundamental period $T_1$, from $T_1/\gamma$ to $\gamma T_1$:
\begin{equation}
\mathrm{IM}_{\mathrm{band}}=\exp\!\left(\frac{1}{N}\sum_{k=1}^{N}\ln S_a^{(5\%)}(T_k)\right), \quad T_k \in [T_1/\gamma,\ \gamma T_1].
\label{eq:IMband}
\end{equation}

The band-averaged geometric mean PSA in Eq.~\eqref{eq:IMband} follows the family of $S_a$-based intensity measures that average logarithmic ordinates over a period band; centering the band on $T_1$ improves efficiency and sufficiency for first-mode dominated shear frames. This is analogous to Sa,avg- or AvSv-type IMs used in record selection and scaling studies \citep{Vargas-Alzate2022}. The Arias window $[t_5,t_{95}]$ is used throughout to avoid start-up and coda artefacts in peak/percentile metrics, consistent with significant-duration definitions based on $D_{5\text{--}95}$ \citep{Davatgari-Tafreshi2023,Jiang2025}.

The scale factor $s=\mathrm{IM}_\star/\mathrm{IM}_{\mathrm{band}}$, clamped to admissible bounds (e.g., [0.2, 2.2]), is applied to obtain the scaled motion $\tilde a_g(t)$. This scaled motion is then used identically for the bare frame and all damper configurations.

All performance metrics (e.g., PFA, IDR) are evaluated over the significant duration of the motion, defined by the 5\%–95\% Arias-intensity window $[t_5, t_{95}]$, to suppress start-up and coda artefacts \citep{Davatgari-Tafreshi2023,Jiang2025}. Per-record metrics are then aggregated (arithmetic means) for the optimization objectives.

\begin{table}[!t]
\centering\footnotesize
\caption{Selected earthquake records used in this study (horizontal components).}
\label{tab:gm-list}
\setlength{\tabcolsep}{6pt}
\renewcommand{\arraystretch}{1.12}
\begin{tabularx}{\linewidth}{@{} 
    l        % Code
    L        % Earthquake name  (esnek, soldan hizalı)
    S[table-format=4.0] % Year
    L        % Station / Component (esnek, soldan hizalı)
    S[table-format=1.1] % Mw
    S[table-format=2.2] % Distance
    S[table-format=1.3] % PGA
@{}}
\toprule
Code & Earthquake name & {Year} & Station / Component & {$M_w$} & {Distance (km)} & {PGA (g)} \\
\midrule
KB95 & Kobe & 1995 & KJMA / KJMA-000                   & 6.9 & 0.96  & 0.833 \\
NR94 & Northridge & 1994 & Sylmar County Hospital / 360 & 6.7 & 9.90  & 0.842 \\
IR80 & Irpinia (Italy) & 1980 & Sturno (STN) / STU270    & 6.9 & 6.78  & 0.320 \\
MJ90 & Manjil & 1990 & Abbar / ABBAR-L                 & 7.3 & 12.55 & 0.510 \\
LP89 & Loma Prieta & 1989 & Calaveras Reservoir / CLR180& 6.9 & 35.28 & 0.110 \\
CC99 & Chi--Chi & 1999 & TCU078 / 90$^\circ$            & 7.6 & 8.30  & 0.442 \\ % 433.6 cm/s^2 → 0.442 g
MR23 & Kahramanmaras (Türkiye) & 2023 & TK~4614 / HNE             & 7.8 & 9.80  & 2.209 \\ % 2166.435 cm/s^2 → 2.209 g
TB78 & Tabas & 1978 & Tabas / H2                        & 7.4 & 2.10  & 0.862 \\
CM92 & Cape Mendocino & 1992 & Petrolia / H2             & 7.0 & 8.20  & 0.662 \\
CH07 & Chuetsu & 2007 & Joetsu Kakizaki / 65010EW      & 6.8 & 9.43  & 0.580 \\
\bottomrule
\end{tabularx}

\vspace{2pt}
{\RaggedRight\footnotesize
Sources: PEER NGA-West2 and AFAD Strong Motion Databases.

Notes: Distances are closest-to-fault (R\textsubscript{rup}) where available. PGA values are in $g$; Chi--Chi and Kahramanmaras are converted from cm/s$^2$ using $1\,g=980.665$ cm/s$^2$ (433.6 $\rightarrow$ 0.442 $g$; 2166.435 $\rightarrow$ 2.209 $g$). Codes refer to internal labels; MR23 corresponds to AFAD record TK~4614 (HNE component), while the remaining records are taken from PEER NGA-West2 with the station/component identifiers shown in the table. The record set intentionally includes high-PGA, near-fault records (e.g., MR23 at \SI{2.209}{g}) to test the model's robustness. These records are characterized by high-velocity pulses, which impose extreme demands on velocity-dependent devices; their inclusion ensures the optimized design remains safe (in terms of pressure, temperature, and cavitation) even under these severe impulsive loads.
\par}
\end{table}
\FloatBarrier

% --- BIBLIOGRAPHY SECTION (REVISED) ---
\bibliographystyle{elsarticle-num}

\IfFileExists{bibtext.bib}{%
  \bibliography{bibtext}%
}{%
  % If 'bibtex.bib' is not found, it prints a placeholder
  \begin{thebibliography}{00}
  \bibitem[{Placeholder(2025)}]{placeholder2025}
  Placeholder Author, ``CRITICAL ERROR: 'bibtext.bib' file not found. Please create it and add BibTeX entries,'' (2025).
  \end{thebibliography}%
}

\end{document}
